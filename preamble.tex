%%%%%%%%%%%%%%%%%%%%%%%%%%%%%%%%%%%%%%%%%%%%%%%%%%%%%%%%%%%%%%%%%%%%%%%%%%%%%%%
%                                                                             %
% Preámbulo del documento                                                     %
%                                                                             %
%%%%%%%%%%%%%%%%%%%%%%%%%%%%%%%%%%%%%%%%%%%%%%%%%%%%%%%%%%%%%%%%%%%%%%%%%%%%%%%

% Paquetes esenciales
%\documentclass[11pt, a4paper]{article}
\usepackage[spanish]{babel}
\usepackage[utf8]{inputenc}
\usepackage[T1]{fontenc}
\usepackage{microtype} % Mejora la tipografía
\usepackage{geometry}
\geometry{left=3cm,right=3cm,top=3cm,bottom=3cm}

% Tipografías modernas
%\usepackage{firamath-otf}
\usepackage[sfdefault]{FiraSans} % Fira Sans como fuente principal
\usepackage{FiraMono} % Fira Mono para código


% Paquetes para imágenes y tablas
\usepackage{graphicx}
\usepackage{tabularx, booktabs}
\usepackage{float}

% Estilos de párrafos
\usepackage{parskip}
\setlength{\parskip}{0.5em}

% Colores y diseño
\usepackage{xcolor}
\definecolor{primary}{RGB}{0, 90, 180} % Azul
\definecolor{secondary}{RGB}{90, 180, 0} % Verde
\usepackage{sectsty}
\allsectionsfont{\sffamily\color{primary}} % Colores para secciones
% Colores para el documento
\definecolor{guindapoli}{RGB}{102, 0, 51}
\definecolor{azulescom}{RGB}{0, 0, 102}
\definecolor{azulclaro}{RGB}{222, 232, 255}
\definecolor{azulfuerte}{RGB}{60, 150, 250}
\definecolor{blanco}{RGB}{255, 255, 255}

% Códigos y listados
\usepackage{listings}
\lstdefinestyle{codeStyle}{
    backgroundcolor=\color{gray!10}, % Fondo gris claro para el código
    commentstyle=\color{green!40!black},
    keywordstyle=\color{primary},
    numberstyle=\tiny\color{gray},
    stringstyle=\color{orange},
    basicstyle=\ttfamily\small,
    breakatwhitespace=false,
    breaklines=true,
    captionpos=b,
    keepspaces=true,
    numbers=left,
    numbersep=5pt,
    showspaces=false,
    showstringspaces=false,
    showtabs=false,
    tabsize=4
}
\lstset{style=codeStyle}

% Enlaces y URLs
\usepackage[hidelinks]{hyperref}
\hypersetup{
    colorlinks=true,
    linkcolor=primary,
    filecolor=magenta,
    urlcolor=blue,
}

% Glosarios y apéndices
\usepackage[acronym,toc]{glossaries}
\makeglossaries
\usepackage[toc,page]{appendix}

% Para mostrar unidades correctamente
\usepackage{siunitx}

