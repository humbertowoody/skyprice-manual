%%%%%%%%%%%%%%%%%%%%%%%%%%%%%%%%%%%%%%%%%%%%%%%%%%%%%%%%%%%%%%%%%%%%%%%%%%%%%%%
%                                                                             %
% 07 - Preguntas frecuentes                                                   %
%                                                                             %
%%%%%%%%%%%%%%%%%%%%%%%%%%%%%%%%%%%%%%%%%%%%%%%%%%%%%%%%%%%%%%%%%%%%%%%%%%%%%%%

\chapter{\textcolor{azulescom}{Preguntas frecuentes}}

\begin{enumerate}
\item \textbf{¿Qué es un modelo de aprendizaje automático?} \\
Un modelo de aprendizaje automático es un algoritmo que aprende de los datos y hace predicciones sobre los datos nuevos. Los modelos de aprendizaje automático se pueden clasificar en dos categorías: supervisados y no supervisados. Los modelos supervisados requieren etiquetas para entrenar, mientras que los modelos no supervisados no requieren etiquetas.
\item \textbf{¿Qué significan las gráficas de ajuste y cómo se interpretan?} \\
Las gráficas de ajuste son una herramienta útil para evaluar el rendimiento de un modelo de aprendizaje automático. Las gráficas de ajuste muestran la relación entre las predicciones del modelo y los valores reales. Si las predicciones del modelo son cercanas a los valores reales, el modelo es preciso. Si las predicciones del modelo son muy diferentes de los valores reales, el modelo es inexacto.
\item \textbf{¿Qué es la validación cruzada y por qué es importante?} \\
La validación cruzada es una técnica utilizada para evaluar el rendimiento de un modelo de aprendizaje automático. La validación cruzada divide los datos en conjuntos de entrenamiento y prueba, y entrena el modelo en el conjunto de entrenamiento y lo evalúa en el conjunto de prueba. La validación cruzada es importante porque ayuda a evitar el sobreajuste del modelo y proporciona una estimación más precisa del rendimiento del modelo.
\item \textbf{¿Qué es el sobreajuste y cómo se puede evitar?} \\
El sobreajuste es un problema común en el aprendizaje automático en el que un modelo se ajusta demasiado a los datos de entrenamiento y no generaliza bien a los datos nuevos. El sobreajuste se puede evitar utilizando técnicas como la validación cruzada, la regularización y la selección de características.
\end{enumerate}


